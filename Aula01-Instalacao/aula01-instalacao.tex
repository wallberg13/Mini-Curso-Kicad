\documentclass{beamer}

\usepackage{verbatim}
\usepackage{graphicx}
\usepackage{booktabs}
\usepackage[brazil]{babel}
\usepackage[utf8x]{inputenc}
\usepackage[portuguese, ruled, linesnumbered]{algorithm2e}
\usepackage{float}
\usetheme{Madrid}
\usepackage{tikz}
\usepackage{listings}
\usepackage{float}
\usepackage{adjustbox}
\usepackage{amsmath}
\usepackage{geometry}
\usepackage{colortbl}

\newcommand{\cinza}{\cellcolor[gray]{0.5}}
\definecolor{gray}{gray}{0.6}


\title[Aula 01 - Instalação]{Aula 01 - Instalação KiCad}

\author[CAECP]{Centro Acadêmico de Engenharia de Computação}

\institute[UTFPR] 
{
  Departamento de Informática - DAINF \\
  Universidade Tecnológica Federal do Paraná
}

\date[\today]{Mini-curso KiCad\\\today}

\begin{document}
\begin{frame}
  \titlepage
\end{frame}

\begin{frame}{Sumário}
  \tableofcontents
\end{frame}

\section{Introdução}
\begin{frame}
		
\end{frame}

\section{Instalação e Download}
\begin{frame}{Instalação e Download}
	\begin{block}{Windows 64 bits}
		\textit{Download} pelo \textit{link}:
		\begin{itemize}
			\item \url{http://downloads.kicad-pcb.org/windows/stable/kicad-5.0.0-x86\_64.exe}
		\end{itemize}
		Depois de baixado, fazer a instalação clássica.
	\end{block}
	\begin{block}{Ubuntu}
		Pelos comandos no terminal:
		\begin{itemize}
			\item \texttt{sudo add-apt-repository --yes ppa:js-reynaud/kicad-5}
			\item \texttt{sudo apt update}
			\item \texttt{sudo apt install kicad}
		\end{itemize}
	\end{block}
\end{frame}

\begin{frame}{Instalação e Download}
	\begin{block}{Arch Linux}
		Pelo comando:
		\begin{itemize}
			\item \texttt{sudo pacman -S kicad}
		\end{itemize}
	\end{block}
\end{frame}

\section{Download de Bibliotecas - Somente Windows}
\begin{frame}{Download de Bibliotecas - Somente Windows}
	Pelos os links:
	\begin{itemize}
		\item Símbolos: \\\url{https://github.com/KiCad/kicad-symbols/archive/master.zip}
		\item \textit{Footprints}: \\\url{https://github.com/KiCad/kicad-footprints/archive/master.zip}
		\item Componentes em 3D \\\url{https://github.com/kicad/kicad-packages3d/archive/master.zip}
	\end{itemize}
\end{frame}

\section{Download de Bibliotecas - Linux}
\begin{frame}{Download de Bibliotecas - Linux}
	\begin{block}{Ubuntu}
		Pelo comando:
		\begin{itemize}
			\item \texttt{sudo apt-get install kicad-}
		\end{itemize}
	\end{block}
	\begin{block}{Arch Linux}
		Pelo comando:
		\begin{itemize}
			\item \texttt{sudo pacman -S kicad-library kicad-library-3d}
		\end{itemize}
	\end{block}
\end{frame}

\begin{frame}
  \titlepage
\end{frame}

\end{document}



